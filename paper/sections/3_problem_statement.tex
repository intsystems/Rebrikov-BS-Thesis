\section{Постановка задачи}

    В данной работе рассматривается задача оптимизации конечной суммы функций следующего вида:
\[
\min_{x \in \mathbb{R}^d} f(x) := \frac{1}{n} \sum_{i=1}^{n} f_i(x),
\]
где каждая функция \( f_i: \mathbb{R}^d \to \mathbb{R} \) соответствует функции потерь на одном элементе выборки. Предполагается, что \( f \) дифференцируема, однако может быть как выпуклой, так и невыпуклой. Основная цель — нахождение точки, минимизирующей функцию \( f \) с использованием стохастических градиентных методов, не требующих вычисления полного градиента на каждой итерации.

Для теоретического анализа вводится ряд стандартных предположений.

\begin{assumption}[Гладкость функций]\label{ass:smoothness}
Каждая функция \( f_i \) обладает \( L \)-гладкостью, то есть для любых \( x, y \in \mathbb{R}^d \) выполняется неравенство:
\[
\|\nabla f_i(x) - \nabla f_i(y)\| \leq L \|x - y\|.
\]
\end{assumption}

\begin{assumption}[Сильная выпуклость]\label{ass:strongconvex}
Каждая функция \( f_i \) является \( \mu \)-сильно выпуклой, то есть для любых \( x, y \in \mathbb{R}^d \) выполняется:
\[
f_i(y) \geq f_i(x) + \langle \nabla f_i(x), y - x \rangle + \frac{\mu}{2} \|y - x\|^2.
\]
\end{assumption}

\begin{assumption}[Невыпуклость]\label{ass:nonconvex}
Функция \( f \) может быть невыпуклой, но при этом обладает конечным инфимумом:
\[
f^* := \inf_{x \in \mathbb{R}^d} f(x) > -\infty.
\]
\end{assumption}
