\section{Заключение}

В данной работе был проведён анализ метода стохастического градиентного типа без полного градиента, основанного на итеративном уточнении оценок градиента. Рассмотренный алгоритм обладает простой структурой обновления.

Были установлены оценки скорости сходимости в двух постановках: невыпуклой и сильно выпуклой. В первом случае подтверждающая эффективность метода при минимальных предположениях о функции. Во втором случае соответствующая классическим результатам для методов первого порядка в сильно выпуклых задачах.

Кроме того, для теоретического обоснования приведены ключевые леммы, использованные при получении оценок. Все выводы сопровождаются строгими доказательствами, а ограничения на шаг метода подобраны таким образом, чтобы обеспечить выполнение условий сходимости при минимальных требованиях к гиперпараметрам.

Полученные результаты подтверждают практическую применимость алгоритма и могут быть использованы в дальнейшем при разработке более сложных схем стохастической оптимизации с пониженной вычислительной сложностью.